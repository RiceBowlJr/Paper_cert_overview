\documentclass[journal, a4paper]{IEEEtran}

\usepackage[utf8x]{inputenc}
\usepackage[english]{babel}

\usepackage{listings}

\usepackage{fancyvrb}

% some very useful LaTeX packages include:

%\usepackage{cite}      % Written by Donald Arseneau
                        % V1.6 and later of IEEEtran pre-defines the format
                        % of the cite.sty package \cite{} output to follow
                        % that of IEEE. Loading the cite package will
                        % result in citation numbers being automatically
                        % sorted and properly "ranged". i.e.,
                        % [1], [9], [2], [7], [5], [6]
                        % (without using cite.sty)
                        % will become:
                        % [1], [2], [5]--[7], [9] (using cite.sty)
                        % cite.sty's \cite will automatically add leading
                        % space, if needed. Use cite.sty's noadjust option
                        % (cite.sty V3.8 and later) if you want to turn this
                        % off. cite.sty is already installed on most LaTeX
                        % systems. The latest version can be obtained at:
                        % http://www.ctan.org/tex-archive/macros/latex/contrib/supported/cite/

\usepackage{graphicx}   % Written by David Carlisle and Sebastian Rahtz
                        % Required if you want graphics, photos, etc.
                        % graphicx.sty is already installed on most LaTeX
                        % systems. The latest version and documentation can
                        % be obtained at:
                        % http://www.ctan.org/tex-archive/macros/latex/required/graphics/
                        % Another good source of documentation is "Using
                        % Imported Graphics in LaTeX2e" by Keith Reckdahl
                        % which can be found as esplatex.ps and epslatex.pdf
                        % at: http://www.ctan.org/tex-archive/info/

%\usepackage{psfrag}    % Written by Craig Barratt, Michael C. Grant,
                        % and David Carlisle
                        % This package allows you to substitute LaTeX
                        % commands for text in imported EPS graphic files.
                        % In this way, LaTeX symbols can be placed into
                        % graphics that have been generated by other
                        % applications. You must use latex->dvips->ps2pdf
                        % workflow (not direct pdf output from pdflatex) if
                        % you wish to use this capability because it works
                        % via some PostScript tricks. Alternatively, the
                        % graphics could be processed as separate files via
                        % psfrag and dvips, then converted to PDF for
                        % inclusion in the main file which uses pdflatex.
                        % Docs are in "The PSfrag System" by Michael C. Grant
                        % and David Carlisle. There is also some information
                        % about using psfrag in "Using Imported Graphics in
                        % LaTeX2e" by Keith Reckdahl which documents the
                        % graphicx package (see above). The psfrag package
                        % and documentation can be obtained at:
                        % http://www.ctan.org/tex-archive/macros/latex/contrib/supported/psfrag/

%\usepackage{subfigure} % Written by Steven Douglas Cochran
                        % This package makes it easy to put subfigures
                        % in your figures. i.e., "figure 1a and 1b"
                        % Docs are in "Using Imported Graphics in LaTeX2e"
                        % by Keith Reckdahl which also documents the graphicx
                        % package (see above). subfigure.sty is already
                        % installed on most LaTeX systems. The latest version
                        % and documentation can be obtained at:
                        % http://www.ctan.org/tex-archive/macros/latex/contrib/supported/subfigure/

\usepackage{url}        % Written by Donald Arseneau
                        % Provides better support for handling and breaking
                        % URLs. url.sty is already installed on most LaTeX
                        % systems. The latest version can be obtained at:
                        % http://www.ctan.org/tex-archive/macros/latex/contrib/other/misc/
                        % Read the url.sty source comments for usage information.

%\usepackage{stfloats}  % Written by Sigitas Tolusis
                        % Gives LaTeX2e the ability to do double column
                        % floats at the bottom of the page as well as the top.
                        % (e.g., "\begin{figure*}[!b]" is not normally
                        % possible in LaTeX2e). This is an invasive package
                        % which rewrites many portions of the LaTeX2e output
                        % routines. It may not work with other packages that
                        % modify the LaTeX2e output routine and/or with other
                        % versions of LaTeX. The latest version and
                        % documentation can be obtained at:
                        % http://www.ctan.org/tex-archive/macros/latex/contrib/supported/sttools/
                        % Documentation is contained in the stfloats.sty
                        % comments as well as in the presfull.pdf file.
                        % Do not use the stfloats baselinefloat ability as
                        % IEEE does not allow \baselineskip to stretch.
                        % Authors submitting work to the IEEE should note
                        % that IEEE rarely uses double column equations and
                        % that authors should try to avoid such use.
                        % Do not be tempted to use the cuted.sty or
                        % midfloat.sty package (by the same author) as IEEE
                        % does not format its papers in such ways.

\usepackage{amsmath}    % From the American Mathematical Society
                        % A popular package that provides many helpful commands
                        % for dealing with mathematics. Note that the AMSmath
                        % package sets \interdisplaylinepenalty to 10000 thus
                        % preventing page breaks from occurring within multiline
                        % equations. Use:
%\interdisplaylinepenalty=2500
                        % after loading amsmath to restore such page breaks
                        % as IEEEtran.cls normally does. amsmath.sty is already
                        % installed on most LaTeX systems. The latest version
                        % and documentation can be obtained at:
                        % http://www.ctan.org/tex-archive/macros/latex/required/amslatex/math/



% Other popular packages for formatting tables and equations include:

%\usepackage{array}
% Frank Mittelbach's and David Carlisle's array.sty which improves the
% LaTeX2e array and tabular environments to provide better appearances and
% additional user controls. array.sty is already installed on most systems.
% The latest version and documentation can be obtained at:
% http://www.ctan.org/tex-archive/macros/latex/required/tools/

% V1.6 of IEEEtran contains the IEEEeqnarray family of commands that can
% be used to generate multiline equations as well as matrices, tables, etc.

% Also of notable interest:
% Scott Pakin's eqparbox package for creating (automatically sized) equal
% width boxes. Available:
% http://www.ctan.org/tex-archive/macros/latex/contrib/supported/eqparbox/

% *** Do not adjust lengths that control margins, column widths, etc. ***
% *** Do not use packages that alter fonts (such as pslatex).         ***
% There should be no need to do such things with IEEEtran.cls V1.6 and later.


% Your document starts here!
\begin{document}

% Define document title and author
	\title{Trustability of certification systems}
	\author{Alexandre Kervadec
	\thanks{Tutor : A.Guermouche}}
	\markboth{University of Bordeaux - Master 2 Computing Sciences}{}
	\maketitle

% Write abstract here
\begin{abstract}
This paper is inspired from E.Gerck "Overview of Certification systems: X.509, CA, PGP and SKIP" \cite{gerck1998overview}. It gives an overview of (main of) certification systems which are X.509 and CAs, PGP, SKIP, DANE and the Certificate transparency by Google. Then it gives a reflection on pros and cons of each system, on a technical point of view and about the government stranglehold on data exchanges.
\end{abstract}

% Each section begins with a \section{title} command
\section{Overview of certification systems}

\subsection{SKIP (Simple Key-Management for Internet Protocol)}

[...]

\subsection{X.509 and CAs}

Description of the different entities

1. CA : can be public (like banks with clients), commercial (like Verisign) or private (like internal departement of a compagny, to log user)
2. Subscriber : sends some infos to the CA to add it to his certificate
3. User : ask infos to CA(s), it's central to the process, since the user party is relying on the informations and is thus at risk

Concerns about the authentification services provided by CAs

1. The content of a certificate needs to be discussed, as well as certificate revocation
2. Issues about DN (Distinguished Name) and CA : a user can possess one or more DN, and use one or more DN on one or more DN
3. Validation of the user, using an ID, which is easily subject to fraud

Going deeper with user validation : DN scheme based on X.500 Recomendation, but it is not completly defined, and will (in 1998) probably not be. Also, X509 certification depends on many others such as ISO, ANSI, ITU and IETF. Thus lead to a lack of harmonization.
Plus, there is a big problem with CPS (Certification Practice Statements), that also can be seen such as flexibility (for pros), because each CA answer specific needs, so no harmonization again.

Some kind of conclusion about harmonization (lack of), in a world wide vision.

\subsection{PGP}

PGP \footnote{Pretty Good Privacy}, created thanks to Phil Zimmermann researches.

[TODO : Schema]

It is based on the \textit{web-of-trust}, if the client doesn't know a CA which gave him a certificate, he ask to the PGP ring.
In that ring, even if the client doesn't know everyone, someone will know him, thus the web-of-trust principle.
Each member of the ring have a trust evaluation of CAs. If the client knows someone he trust that trust the CA he doesn't know, then he accept the certificate and trust the CA.
But if the client trust someone who doesn't trust the CA, he reject the CA.
The big point is when nobody knows the CA, and when trusted friends of the client have a medium evaluation of the CA : when to trust, when to reject, at which level of trust?

\subsection{DANE}

DANE\footnote{DNS-Based Authentication of Named Entities} is not implemented yet, but the IETF is working on the standard.

\subsection{Cetificate Transparency}

Because of a case of Google CA's usurpation, Google is working on a new system that can improve certification systems.

\nocite{*}

\begin{thebibliography}{99}

\bibitem{gerck1998overview} E.Gerck, Overview of Certification Systems: X. 509, CA, PGP and SKIP, 1998, http://www.blackhat.com/presentations/bh-usa-99/EdGerck/certover.pdf

\bibitem{skip2010user} Oracle, SunScreen SKIP User's Guide, Release 1.1, 2010, http://docs.oracle.com/cd/E19957-01/805-5743/6j5dvnrfs/index.html

\bibitem{dane2012} P Hoffman, J Schlyter, The DNS-based authentication of named entities (DANE) transport layer security (TLS) protocol: TLSA, 2012, https://www.rfc-editor.org/rfc/pdfrfc/rfc6698.txt.pdf

\end{thebibliography}

% Your document ends here!
\end{document}
